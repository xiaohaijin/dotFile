\chapter{linux配置}

\section{CMake安装和常用库的CMakeLists.txt的写法}
CMake是一个跨平台的生成工具,%
主要使用平台无关的CMakeLists.txt生成平台相关的Makefile文件。%

\noindent\textbf{\color{magenta}CMake的安装:}
\begin{bashcode}
tar -zvxf cmake.tar.gz
cd cmake
./configure --prefix=to/install/path --qt-gui
make -j4
make install
\end{bashcode}


\subsection{Boost}
\noindent\textbf{\color{magenta}Boost库的CMakeList.txt的写法:}
\begin{cmakecode}
  cmake_minimum_required(VERSION 3.0)

  set(CMAKE_PREFIX_PATH "/opt/boost/install")
  set(Boost_USE_STATIC_LIBS OFF)
  set(Boost_USE_MULTITHREADED ON)
  set(Boost_STATIC_RUNTIME OFF)

  find_package(Boost)
  include_directories(\${Boost_INCLUDE_DIRS})

  message(STATUS \${Boost_INCLUDE_DIRS})
  message(STATUS \${Boost_LIBRARIES})

  add_subdirectory(src)
\end{cmakecode}
其中Boost\_INCLUDE\_DIRS必须显示的添加到头文件的搜索路径中。

\subsection{OpenCV}
\noindent\textbf{\color{magenta}OpenCV库的CMakeLists.txt的写法:}
\begin{cmakecode}
  file(GLOB SRC_FILES ${CMAKE_CURRENT_SOURCE_DIR}/*.cpp)

  # Declare the executable target built from your sources                        
  add_executable(${PROJECT_NAME} ${SRC_FILES})

  # Link your application with OpenCV libraries           
  target_link_libraries(${PROJECT_NAME} ${OpenCV_LIBS})

  # 安装
  install(TARGETS ${PROJECT_NAME}
          RUNTIME DESTINATION ${PROJECT_SOURCE_DIR}/bin)
\end{cmakecode}
注意OpenCV的库的搜索结果是放在OpenCV\_LIBS这个变量中的,%
且头文件无需添加到搜索路径中。%
最好的行是就是看下OpenCV库自带例子的写法。

\subsection{ROOT}
\noindent\textbf{\color{magenta}ROOT库的CMakeLists.txt的写法:}
\begin{cmakecode}
  # 生成files
  file(GLOB SRC "${PROJECT_SOURCE_DIR}/src/*.cpp")

  # 添加目标文件
  add_executable(${PROJECT_NAME} ${SRC})

  # 添加依赖库
  target_link_libraries(${PROJECT_NAME} ${ROOT_LIBRARIES})

  # 安装
  install(TARGETS ${PROJECT_NAME}
          RUNTIME DESTINATION ${PROJECT_SOURCE_DIR}/bin)
\end{cmakecode}

注意ROOT的库的搜索结果是放在ROOT\_LIBRARIES这个变量中的,%
且头文件无需添加到搜索路径中。%

\subsection{Qt5}
\noindent\textbf{\color{magenta}Qt5库的CMakeLists.txt的写法:}
\begin{cmakecode}
  # 最低需要2.8.11版本
  cmake_minimum_required(VERSION 3.0)

  # 设定Qt的安装位置
  set(CMAKE_PREFIX_PATH "/opt/qt/install/")

  # include的目录包含当前目录
  set(CMAKE_INCLUDE_CURRENT_DIR ON)

  # 工程名字
  project(QtUseCMake)

  # 查找需要的Qt库文件
  find_package(Qt5Widgets REQUIRED)
  find_package(Qt5Core REQUIRED)
  find_package(Qt5Gui REQUIRED)

  # 设定头文件和UI的搜索路径
  set(HEADER_DIR "${CMAKE_CURRENT_SOURCE_DIR}/include")
  set(GUI_DIR "${CMAKE_CURRENT_SOURCE_DIR}/ui")
  set(RESOURCES_DIR "${CMAKE_CURRENT_SOURCE_DIR}/resources")

  # 添加头文件的搜索路径
  include_directories(${HEADER_DIR})

  # 添加子目录
  add_subdirectory(src)

  # 安装
  install(FILES COPYRIGHT README.md
          DESTINATION doc)
\end{cmakecode}
其中src目录下的CMakeLists.txt的写法如下:
\begin{cmakecode}
  # 生成files
  file(GLOB SRC "${PROJECT_SOURCE_DIR}/src/*.cpp")
  file(GLOB MOC_HEADER "${PROJECT_SOURCE_DIR}/include/*.h")
  file(GLOB UIC_UI "${PROJECT_SOURCE_DIR}/ui/*.ui")
  file(GLOB QRC "${PROJECT_SOURCE_DIR}/resources/action.qrc")

  # 通过UI文件生成对应的头文件
  qt5_wrap_cpp(MOC_SRC ${MOC_HEADER})
  qt5_wrap_ui(UIC_SRC ${UIC_UI})
  qt5_add_resources(QRC_SRC ${QRC})

  # 添加目标文件
  add_executable(${PROJECT_NAME} ${SRC} ${MOC_SRC} ${UIC_SRC} ${QRC_SRC})

  # 添加依赖库
  target_link_libraries(${PROJECT_NAME} Qt5::Widgets Qt5::Core Qt5::Gui)

  # 安装
  install(TARGETS ${PROJECT_NAME}
          RUNTIME DESTINATION bin)
\end{cmakecode}
库的依赖方式为target\_link\_libraries(target Qt5::Core Qt5::Widgets Qt5::Gui Qt5::其它模块)。





\section{OpenCV的安装}
这里我们主要给出Ubuntu下OpenCV的安装例子。%
首先我们要先安装OpenCV所需要的依赖,%
打开terminal输入以下的命令:
\begin{bashcode}
  sudo apt-get -y install libopencv-dev build-essential cmake libdc1394-22 libdc1394-22-dev libjpeg-dev libpng12-dev libtiff-dev libjasper-dev libavcodec-dev libavformat-dev libswscale-dev libxine-dev libgstreamer0.10-dev libgstreamer-plugins-base0.10-dev libv41-dev libtbb-dev libqt4-dev libmp3lame-dev libopencore-amrnb-dev libopencore-amrwb-dev libtheora-dev libvorbis-dev libxvidcore-dev x264 v41-utils
\end{bashcode}

接着我们从官网或者github下载自己需要版本的源代码,%
并且下载opencv\_contrib的测试模块。%
分别解压缩两个源代码。%
\begin{bashcode}
  cd opencv-版本
  mkdir build
  cd build
  cmake -D CMAKE_BUILD_TYPE=RELEASE -D CMAKE_INSTALL_PREFIX=/full/path/to/opencv/install/dir -D INSTALL_C_EXAMPLES=ON -D BUILD_EXAMPLES=ON -D OPENCV_EXTRA_MODULES_PATH=/full/path/to/opencv-contrib/modules ../
  make -j4
  sudo make install
\end{bashcode}
 剩下只需要修改下PATH和LD\_LIBRARY\_PATH环境变量的值。


\section{GNOME主题桌面配置}
一个好看的GNOME桌面可以使做事情时感到赏心悦目,%
因此倒腾一个好看的桌面主题是势在必然的。%
各种下载主题配置的工具的网址肯定是gnome的官网了。%
这里只记录下我用了什么主题:\\
\noindent{}\textbf{\color{magenta}Appearance的控制:}
\begin{itemize}
\item{GTK+:Sierra-compact-light-thin}
\item{Icons:Flat-Remix-Light}
\item{Cursor:Adwaita(default)}
\item{Shell theme:Sierra-compact-light-thin}
\item{Enable animations:打开}
\end{itemize}

\noindent{}\textbf{\color{magenta}Desktop的控制:}
\begin{itemize}
\item{Mode:两个模式全部的打开}
\item{墙纸:这个这个看着找吧}
\end{itemize}

\noindent{}\textbf{\color{magenta}Extensions:}
\begin{itemize}
\item{打开User themes即可,剩下的自己看着办吧。}
\end{itemize}

\noindent{}\textbf{\color{magenta}Fonts:}
\begin{itemize}
\item{Window Titles : Cantarell Bold 13}
\item{Interface : Cantarell Regular 13}
\item{Monospace : Monospace Regular 13}
\item{Hinting : Slight}
\item{Antialiasing : Grayscale}
\item{Scaling Factor : 1.00}
\end{itemize}

\noindent{}\textbf{\color{magenta}Keyboard and Mouse}
\begin{itemize}
\item{Show All Input Sources : close}
\item{Key theme : Emacs}
\item{Switch between overview and desktop : Left super}
\item{Acceleration profile : Flat}
\item{Show location of pointer : open}
\item{Middle-click Paste : open}
\item{Click method : default}
\end{itemize}

\noindent{}\textbf{\color{magenta}Power}
\begin{itemize}
\item{Action : Suspend}
\item{Don't suspend on lid close : close}
\end{itemize}

\noindent{}\textbf{\color{magenta}Top Bar}
\begin{itemize}
\item{Show Application Menu : on}
\item{Show date : checked}
\item{Show seconds : checked}
\item{Show week numbers : checked}
\end{itemize}







%%% Local Variables:
%%% mode: latex
%%% TeX-master: t
%%% End:
